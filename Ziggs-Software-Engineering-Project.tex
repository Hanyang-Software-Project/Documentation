\documentclass[conference]{IEEEtran}
\IEEEoverridecommandlockouts
\usepackage{enumitem}
\usepackage{cite}
\usepackage{amsmath,amssymb,amsfonts}
\usepackage{algorithmic}
\usepackage{graphicx}
\usepackage{textcomp}
\usepackage{xcolor}
\usepackage{multicol}
\usepackage{hyperref}
\usepackage{float}
\usepackage{array}
\usepackage{adjustbox}
\usepackage{fancyhdr}
\pagestyle{fancy}
\fancyhf{}
\fancyfoot[C]{\thepage}
\renewcommand{\headrulewidth}{0pt}
\renewcommand{\footrulewidth}{0pt}

\def\BibTeX{{\rm B\kern-.05em{\sc i\kern-.025em b}\kern-.08em T\kern-.1667em\lower.7ex\hbox{E}\kern-.125emX}}
\begin{document}

\title{ZIGGS - Sound detection in households}

\author{ 
\IEEEauthorblockN{Jan Imhof} 
\IEEEauthorblockA{\textit{Computer Science Dep.} \\ 
\textit{Karlsruhe Institute of Technology}\\ 
Karlsruhe, Germany \\ 
uwxdb@student.kit.edu}
\and 
\IEEEauthorblockN{Ivan Milosavljevic} 
\IEEEauthorblockA{\textit{Computer Science Dep.} \\ 
\textit{ISEP School of Electronics}\\ 
Boulogne-Billancourt, France \\ 
ivmi61663@eleve.isep.fr}
\and 
\IEEEauthorblockN{Nicolas Busquet} 
\IEEEauthorblockA{\textit{Computer Science Dep.} \\ 
\textit{ESGI School of CS}\\ 
Nanterre, France \\ 
nrb.busquet@gmail.com}
\and 
\IEEEauthorblockN{Adrian Obermuehlner} 
\IEEEauthorblockA{\textit{Computer Science Dep.} \\ 
\textit{ZHAW Zurich University}\\ 
Zurich, Switzerland \\ 
aobermuhlner@gmail.com}
}
\maketitle



\maketitle

\section{Introduction}
\subsection{Motivation}
In today’s fast-paced world, ensuring the safety and security of our homes and loved ones has become more critical than ever. Whether it is monitoring elderly family members living independently, ensuring the safety of children, or keeping track of pets when left alone, there is a growing demand for intelligent home monitoring solutions that go beyond basic video surveillance. Traditional security systems often fail to detect nuanced or specific audio-based events such as sudden loud noises, breaking glass, or periods of unusual silence, which can be indicators of emergencies or safety concerns.

The development of an acoustic abnormality detection system utilizing machine learning and IoT devices fills this crucial gap in the market. This innovative software will enable continuous monitoring of environmental sounds through LG's IoT devices, which are already well-integrated into modern smart homes. The use of machine learning allows the system to detect and analyze abnormal audio patterns, such as an elderly person calling for help, a child in distress, or even a pet knocking over an object. Unlike traditional static systems, the machine learning component will learn and improve from new sound data recorded within each household, tailoring its response to the specific acoustic environment of every home.

By integrating this solution with LG's ecosystem of IoT devices, users will have access to a comprehensive, real-time sound monitoring system that adapts to their unique needs. The system will send timely alerts to their mobile devices, allowing them to respond immediately to any potentially dangerous or unusual situations. This adaptive learning capability, combined with the flexibility of using various LG IoT devices such as smart speakers, cameras, or even home appliances, ensures that the system provides an enhanced, personalized safety net for users.

The primary use cases for this system include elderly care, where rapid response to abnormal sounds could prevent accidents or health emergencies; child safety monitoring, where sudden loud noises or periods of unusual silence could indicate a problem; and pet monitoring, where the system can alert owners to disturbances caused by their pets. As homes become more connected, the need for advanced audio-based monitoring systems is increasingly evident, making this software a valuable addition to the market.

\subsection{Problem Statement}
Current home monitoring systems fail to effectively detect crucial audio cues like breaking glass or sudden loud noises, often resulting in missed events or frequent false alarms. These systems also lack the ability to adapt to the unique sound environments of individual households, making them unreliable for families with elderly members, children, or pets.

Our solution addresses this by using machine learning and IoT devices to continuously monitor sounds and detect abnormalities. Through user feedback, the system will refine its model over time, adapting to each household and reducing false alarms, ensuring timely and accurate alerts.

\subsection{Related Software}
\textbf{SimpliSafe}: This is a well-known home security app that pairs with its proprietary camera and detection systems. While primarily focused on video monitoring, SimpliSafe uses various sensors, such as motion detectors and entry sensors, to identify potential intrusions. Sound detection is used in a specialized way, mainly for detecting the sound of breaking glass, which is a common indicator of a break-in.

\textbf{ASTRA}: This app is described as “state of the art” for anomaly detection, with a focus on video data. ASTRA uses advanced algorithms and artificial intelligence to identify unusual activities through video feeds, which can include identifying abnormal movements or behaviors. Its strength lies in video anomaly detection, making it suitable for environments where visual analysis is more critical than audio.

\textbf{Edge Impulse}: Edge Impulse provides a platform that integrates machine learning with IoT devices, focusing on real-time anomaly detection for environmental and acoustic monitoring. Their system is highly adaptable and has been deployed in various use cases, from smart home appliances to wearables. They emphasize ease of use, allowing developers to collect audio data, train models, and deploy them on IoT devices for real-time detection.

\textbf{Echo-Guard}: This system focuses on detecting anomalous sounds in smart manufacturing environments. It uses acoustic sensors combined with machine learning to monitor the sounds produced by machinery and detect abnormal events, such as equipment failure or unexpected noises. Its use of convolutional neural networks (CNN) helps to convert audio data into spectrograms, enabling the system to classify and detect anomalies in real-time.

\section{Role Assignments}
\begin{table}[H]
    \centering
    \caption{Role Assignments}
    \begin{adjustbox}{max width=\textwidth}
        \begin{tabular}{|>{\centering\arraybackslash}m{1.2cm}|>{\centering\arraybackslash}m{1.6cm}|m{4.8cm}|}
        \hline
        \textbf{Role} & \textbf{Name} & \textbf{Task Description} \\
        \hline
        AI Developer & Jan Imhof & Specializes in designing and optimizing machine learning algorithms to identify abnormal sounds in a home environment. Responsible for training the model using diverse audio data, allowing it to recognize and adapt to different household sound profiles, such as elderly distress calls, child-related noises, or pet disturbances. Ensures continuous learning from new data, refining accuracy and effectiveness in detecting potential emergencies. \\
        \hline
        Back-end Developer & Ivan Milosavljevic & Responsible for the development and integration of the back-end architecture that connects LG’s IoT devices with the acoustic monitoring system. This includes managing the database to store audio data, implementing security protocols for user authentication, and building APIs for smooth data transmission and real-time alerts. Oversees scalability to handle continuous audio streams and machine learning updates. \\
        \hline
        Front-end Developer & Adrian Samuel & Focuses on designing and developing the user interface (UI) and user experience (UX) for the mobile and web platforms that enable users to monitor sound-based events in their homes. Ensures seamless interaction between users and the system, including configuring alerts and viewing real-time audio analysis. Key tasks include testing, debugging, and refining the interface for ease of use across various devices. \\
        \hline
        Front-end Developer & Nicolas Busquet & Oversees the visual design and interaction elements of the user-facing components, ensuring that users receive notifications and alerts in a visually engaging and intuitive manner. Works with tools like Figma to create clean, responsive designs that facilitate quick responses to detected abnormalities, tailored for users with specific needs such as elderly care or pet monitoring. \\
        \hline
    \end{tabular}
    \end{adjustbox}
\end{table}

\section{Requirements Analysis}
\subsection{User Account Management}
\subsubsection{Sign Up}
\textbf{Context:} The sign-up screen serves as the main entry point for new users to create their accounts. The login functionality is integrated into the same screen, allowing both new registrations and returning users to access their accounts seamlessly.

\textbf{Information Required for Sign Up:}
\begin{itemize}
    \item \textbf{Email:} The user’s email address will be used for sending alerts, notifications, and other important communications. It is also necessary for account recovery and validation.
    \item \textbf{ID/Username:} Users must create a unique identifier for their account, up to 12 characters long. This ID is used to recognize the user within the system and provides an alternative to using their email for login.
    \item \textbf{Password:} Users must create a secure password that meets the following criteria:
    \begin{itemize}
        \item Minimum of 8 characters
        \item Includes lower-case and upper-case letters
        \item Contains numbers and symbols to enhance security
        \item Passwords are encrypted using SHA256 to protect against data breaches and ensure that login data is not stored in plain text.
    \end{itemize}
    \item \textbf{Personal Information:} Users are required to provide their first and last names, phone number, emergency phone number, and postal address. This information helps ensure that in the case of an alert that isn’t acknowledged promptly, emergency contacts can be reached.
\end{itemize}


\textbf{Redirection:}
\begin{itemize}
    \item Upon successful registration, users are automatically redirected to the login page to access their account using their new credentials.
\end{itemize}

\subsubsection{Login}
\textbf{Context:} The login page is the main gateway for returning users to access their accounts and monitor their home environment. It ensures that only authorized users can access the system.

\textbf{Required Information for Login:}
\begin{itemize}
    \item \textbf{Email or Username/ID:} Users can log in using either their registered email or their unique username.
    \item \textbf{Password:} The user’s password is required for authentication. The system verifies the encrypted password stored in the database.
\end{itemize}


\subsubsection{Administrator Back-office}
The website is maintained by admin users. These admin users have access to the back-office part of the website from where they can manage the different items of it.



\subsection{User Dashboard}
Once logged in, users have access to a personalized dashboard where they can monitor alerts, manage their account information, and interact with the system.

\textbf{Alerts History:}
\begin{itemize}
    \item Users can view all alerts that have been triggered by the system. These are presented in a tabular format for easy review.
    \item \textbf{Layout:} The table includes columns for the date, time, type of alert, and a brief description of the event. Users can sort or filter alerts based on specific dates, types of events, or alert status (e.g., resolved or pending).
    \item Users can click on any alert to see more details, such as the audio snippet that triggered the alert, and provide feedback if the alert was accurate or a false positive.
\end{itemize}



\subsection{Alerts}

\textbf{Website Notifications:} Alerts are displayed directly on the user’s dedicated dashboard on the website.. Users can review the alert details and access a recorded snippet of the audio to quickly assess the situation.

\textbf{Mandatory User Response:} Users are required to mark each alert as either true or false. If no action is taken within a predefined time frame, the alert will continue to generate warnings, ensuring the issue is not overlooked. This mechanism ensures accountability and timely resolution, particularly in sensitive scenarios like elderly care or child monitoring.


\subsection{Dataset Requirements}
\subsubsection{Normal Data Requirements}
The model requires a dataset that captures typical sounds in child play environments to accurately represent normal conditions.

\begin{itemize}
    \item The dataset should include audio from natural, unstructured settings where children are playing, such as kindergartens, homes, or therapy centers.
    \item The sounds should cover a range of common activities and interactions, reflecting realistic soundscapes in child play environments.
    \item Audio quality should be sufficient to capture subtle environmental sounds without excessive background noise.
    \item The dataset selected for these requirements is the \href{https://www.idiap.ch/en/dataset/childplay-gaze}{ChildPlayGaze Dataset}, which provides realistic audio accompanying children’s play and social interactions.
\end{itemize}

\subsubsection{Abnormal Data Requirements}
To train the model to recognize unusual or potentially hazardous sounds, the dataset must include a variety of abnormal audio events.

\begin{itemize}
    \item Abnormal sound events should include loud noises such as crashes, glass breaking, and other unexpected sounds that deviate from typical child play environments.
    \item These sound events should be clearly labeled to facilitate both supervised and unsupervised anomaly detection.
    \item Data should be varied to represent a wide range of potential abnormal events, helping the model generalize to new or unknown anomalies.
\end{itemize}

\subsubsection{Preprocessing Requirements}
Before using the dataset for model training, audio data must be preprocessed to standardize inputs and improve model performance.

\begin{enumerate}
    \item \textbf{Audio Trimming}
    \begin{itemize}
        \item Audio clips should be limited to a uniform length (e.g., 5 seconds) to ensure consistent input dimensions.
        \item Trimming is essential to focus on the relevant parts of each sound event and reduce variability in input length.
    \end{itemize}

    \item \textbf{Mel Spectrogram Conversion}
    \begin{itemize}
        \item Audio clips must be converted into mel spectrograms, which represent the frequency content of the sound in a way that aligns with human auditory perception.
        \item The spectrograms should be stored in a format suitable for machine learning models, such as \texttt{.npy} files.
        \item Parameters for mel spectrogram generation should be standardized across all audio files to maintain consistency in the input features.
    \end{itemize}
\end{enumerate}

\section{System Specifications}
\subsection{Supervised Learning Requirements}
\begin{itemize}
    \item The model must be able to classify sounds based on labeled data.
    \item It should distinguish between normal and abnormal sounds in various environments.
    \item The system is expected to identify abnormal sounds, such as breaking glass or loud noises, and categorize them accordingly.
\end{itemize}

\subsection{Unsupervised Learning Requirements}
\begin{itemize}
    \item The model should learn typical patterns in normal sound environments without relying on explicit labeling.
    \item It must be able to detect deviations from these normal patterns as potential anomalies.
    \item The system should recognize unfamiliar or novel sounds that may indicate abnormal conditions.
\end{itemize}

\subsection{Model Personalization and Feedback}
\begin{itemize}
    \item After deployment, feedback from users (e.g., parents) on anomalies will be collected.
    \item This feedback will help the model refine its detection capabilities.
    \item The system will be personalized to each household's unique sound environment.
\end{itemize}

\section{Design and Development Environment}
\begin{table}[H]
    \centering
    \caption{Design and Development Environment}
    \begin{adjustbox}{max width=\textwidth}
        \begin{tabular}{|>{\centering\arraybackslash}m{1.2cm}|>{\centering\arraybackslash}m{1.6cm}|m{4.8cm}|}
        \hline
        \textbf{Technology} & \textbf{Usage} & \textbf{Justification} \\
        \hline
        Google Colab & AI Model Development & Provides a cloud-based environment with free access to an NVIDIA Tesla T4 GPU. Allows easy storage and sharing of audio data and model outputs via Google Drive. \\
        \hline
        Python (PyTorch) & AI Programming & Used for developing the AI model due to its extensive libraries and compatibility with PyTorch, which is preferred for modern AI practices. Ensures flexibility for both preprocessing tasks and model training. \\
        \hline
        AWS Lambda & AI Model Deployment & A serverless platform for deploying the AI model. It allows on-demand execution, reducing costs and avoiding the need for continuous server management. Docker is used to package the model and dependencies for consistent runtime behavior. \\
        \hline
        AWS EC2 (Linux) & Front-end and Back-end Hosting & Both the front-end and back-end are deployed on AWS EC2 instances. Linux is chosen for its lightweight environment, efficiency, and compatibility with Dockerized applications. \\
        \hline
        Docker & Deployment & Used to containerize the AI model for AWS Lambda and to deploy the front-end and back-end on AWS EC2. Simplifies deployment and ensures consistency between environments. \\
        \hline
        Spring Boot 3 (Java) & Back-end Development & A popular framework with extensive documentation and tools, aligning with the team's expertise, simplifying development and maintenance. \\
        \hline
        Vue.js 3 (TypeScript) & Front-end Development & Lightweight and fast framework for responsive UIs, supports real-time data binding for instant alerts, and provides a maintainable and scalable codebase using TypeScript. \\
        \hline
    \end{tabular}
    \end{adjustbox}
\end{table}





\section{Cost Estimation}
\subsection{AI Model Development}
This project utilizes free resources provided by Google Colab and Google Drive:
\begin{itemize}
    \item \textbf{Software Cost}: Free, as Google Colab and Google Drive offer no-cost access for the project’s requirements.
    \item \textbf{Hardware Cost}: None, as the Tesla T4 GPU is provided within Google Colab’s free tier.
\end{itemize}

\subsection{AI Model Deployment} 
For this serverless deployment, AWS charges are based on the compute time used per request in Lambda and the request volume handled by API Gateway: 
\begin{itemize} 
    \item AWS Lambda with Docker: The cost is based on compute time (measured in milliseconds) per request. Given the lightweight nature of the model and expected low frequency of a few calls per day, Lambda expenses are anticipated to be minimal, approximately 0.50–1 Dollar per month. 
    \item API Gateway: This service charges per million requests. With a low request frequency, monthly costs for API Gateway are expected to be negligible, estimated at around 0.10–0.50 Dollar per month. 
    \item Total Cost: The combined costs for AWS Lambda and API Gateway for a few daily calls are estimated at approximately 0.60–1.50 Dollar per month. 
\end{itemize}

\subsection{Application Deployment on AWS EC2}
This project deploys the front-end and back-end on AWS EC2 instances:
\begin{itemize}
    \item \textbf{EC2 Instance Cost}: Using a t3.small instance for both the front-end and back-end is sufficient for the expected traffic. The monthly cost is approximately 3 Dollars per instance, depending on usage and data transfer.
    \item \textbf{Storage Cost}: The required EBS storage for each instance is minimal (around 20GB). The estimated cost is approximately 1–2 Dollars per month.
    \item \textbf{Total Cost}: Deploying both instances (front-end and back-end) is estimated at 10 dollars per month, including data transfer and storage.
\end{itemize}

\section{Implementation and Integration}

\subsection{Login System}
The login system allows users to securely access their accounts using their email and password. Users can also opt for a password recovery mechanism through email verification. Spring Boot handles back-end processes, including password encryption using SHA-256 and validation against the database. Vue.js ensures the front-end provides an intuitive and seamless experience for users, with error handling and clear feedback messages on incorrect login attempts or successful authentication.

\subsection{User Dashboard}
The dashboard gives users access to view historical alerts, manage personal information, and configure system settings. It features filters to sort alerts by type and date, allowing users to analyze past events. The front-end, developed with Vue.js, ensures responsive design across devices. Spring Boot manages the back-end, ensuring secure communication and data retrieval from the database.

\subsection{Alerts}
The system sends alerts to the front-end when abnormal sounds are detected. Users can view and manage these alerts through the interface, which allows them to respond appropriately to the detected events.

\subsection{Real-time Sound Monitoring}
The back-end, built using Spring Boot, communicates with the machine learning models, which detect abnormal sounds such as breaking glass, loud noises, or sudden silence. Python models, integrated through APIs, perform sound analysis and anomaly detection, ensuring efficient real-time monitoring.

\subsection{Machine Learning Integration}
Machine learning models play a vital role in anomaly detection. Supervised models identify patterns based on pre-labeled audio data, while unsupervised learning models recognize new and unusual sounds by learning typical household environments. These models are trained using Python and integrated into the system via APIs provided by the Spring Boot back-end.

\subsection{Database Management}
The system uses PostgreSQL to store user information, alert history, and sound data. Spring Boot handles database communication, ensuring data integrity and security. User profiles, system alerts, and other data are stored relationally, facilitating efficient querying and data retrieval. Regular backups and encryption protocols ensure data safety and compliance with privacy policies.

\subsection{Data Implementation Process}
This section details the step-by-step implementation of the audio preprocessing.

\subsubsection{Audio Data Collection and Duration Normalization}
The audio data preprocessing begins with loading WAV files from a designated directory. The audio files are processed to ensure each file meets the required duration of exactly 5 seconds. If a file is shorter than 5 seconds, additional audio files are concatenated to reach the duration limit.

\begin{itemize}
    \item \textbf{Identify WAV files}: List all WAV files in the target directory, excluding specified subfolders.
    \item \textbf{Calculate Duration}: For each audio file, calculate the duration in seconds to determine if it meets the 5-second requirement.
    \item \textbf{Combine and Trim Audio Files}:
        \begin{itemize}
            \item Concatenate audio files until the total duration is at least 5 seconds.
            \item Trim the combined audio to exactly 5 seconds if it exceeds the limit.
            \item Save the trimmed file in a separate output directory, preserving folder structure.
        \end{itemize}
\end{itemize}

\subsubsection{Mel Spectrogram Generation}
Each trimmed audio file is converted into a mel spectrogram, which provides a time-frequency representation of the sound.

\begin{itemize}
    \item \textbf{Load Audio Files}: Load each WAV file from the output directory created in the previous step.
    \item \textbf{Convert to Mel Spectrogram}:
        \begin{itemize}
            \item Use a set of predefined parameters, such as FFT size, hop length, and number of mel bands, to create a standardized mel spectrogram.
            \item Convert the spectrogram to decibel scale to enhance feature visibility.
        \end{itemize}
    \item \textbf{Padding and Trimming}: If the mel spectrogram has fewer frames than expected for a 5-second clip, pad it with zeros. If it exceeds the expected frame count, trim the excess frames to maintain uniform input dimensions.
    \item \textbf{Save as .npy Files}: Save each spectrogram as an \texttt{.npy} file in a structured directory to facilitate easy access during training.
\end{itemize}

\subsection{AI Implementation Process}
The following section describes the main stages involved in implementing the AI model for audio anomaly detection under the development environment on Google Colab with Python and PyTorch.

\subsubsection{Data Preparation and Loading}
\begin{itemize}
    \item \textbf{Data Collection}: The mel spectrograms for the audio files are stored in Google Drive and are structured into folders based on the type of sound (e.g., normal or anomaly).
    \item \textbf{Dataframe Creation}: A function is used to load the mel spectrograms and create a structured dataframe, labeling each entry based on sound type and directory (train or test). The data is split so that 15\% of samples are used for testing, and the rest are used for training.
\end{itemize}

\subsubsection{Data Generator}
The data generator function is responsible for loading data in batches during training and validation:
\begin{itemize}
    \item \textbf{Shuffling and Batch Creation}: To avoid overfitting and ensure variety, data is shuffled and loaded in small batches. Each batch contains both the mel spectrograms and their corresponding labels.
    \item \textbf{One-Hot Encoding of Labels}: Labels are converted to a one-hot encoded format to support multi-class classification.
\end{itemize}

\subsubsection{Model Architecture (CNN with Regularization)}
The model is implemented as a Convolutional Neural Network (CNN) with regularization techniques such as dropout and batch normalization to improve generalization. The main components are:
\begin{itemize}
    \item \textbf{Convolutional Layers}: Extract spatial features from the mel spectrograms. Each convolutional layer is followed by batch normalization and pooling.
    \item \textbf{Fully Connected Layer with Dropout}: After flattening, a fully connected layer with dropout is used for additional regularization.
    \item \textbf{Output Layer}: A softmax activation function is applied in the final layer to classify sounds into six categories.
\end{itemize}

\subsubsection{Training and Validation Loop}
The training function, \texttt{fit\_and\_predict\_model}, manages the training and validation of the model:
\begin{itemize}
    \item \textbf{Data Splitting}: The function divides the data into training and validation sets, using 80\% for training and 20\% for validation within the training dataset.
    \item \textbf{Loss Function and Optimization}: The model is trained using Cross-Entropy Loss for multi-class classification, and the optimizer is Adam with L2 regularization.
    \item \textbf{Training Epochs and Early Stopping}: The model is trained over multiple epochs, and early stopping is used to prevent overfitting by monitoring validation loss.
\end{itemize}


\begin{figure}[h!]
    \centering
    \includegraphics[width=0.5\textwidth]{Model.png}
    \caption{Training and Validation Accuracy and Loss Over Epochs}
    \label{fig:loss_plot}
\end{figure}

\subsection{Model Evaluation}
\begin{table}[h!]
    \centering
    \caption{Classification Report for Machine Learning Model on Various Audio Categories}
    \begin{tabular}{|c|c|c|c|c|}
        \hline
        & \textbf{Precision} & \textbf{Recall} & \textbf{F1-Score} & \textbf{Support} \\
        \hline
        ChildrenPlay & 1.00 & 1.00 & 1.00 & 29 \\
        Accidents & 0.00 & 0.00 & 0.00 & 1 \\
        Car Crash & 0.80 & 1.00 & 0.89 & 4 \\
        Conversation & 1.00 & 1.00 & 1.00 & 2 \\
        Gun Shot & 0.90 & 1.00 & 0.95 & 9 \\
        Scream & 1.00 & 0.83 & 0.91 & 6 \\
        \hline
        Accuracy & \multicolumn{3}{c|}{0.96} & 51 \\
        \hline
        Macro Avg & 0.78 & 0.81 & 0.79 & 51 \\
        Weighted Avg & 0.95 & 0.96 & 0.95 & 51 \\
        \hline
    \end{tabular}
    \label{tab:classification_report}
\end{table}

\noindent
\textbf{Explanation of Terms:}
\begin{itemize}
    \item \textbf{Macro Average}: The macro average calculates precision, recall, and F1-score for each class independently and then takes their unweighted mean. This gives equal importance to each class, which can highlight performance on underrepresented classes.
    \item \textbf{Weighted Average}: The weighted average calculates metrics for each class and then takes their average, weighted by the number of samples (support) in each class. This provides a more accurate overall metric when there is class imbalance.
    \item \textbf{Support}: Support represents the number of true instances for each class in the dataset. It shows the distribution of data across different categories and helps in understanding how prevalent each class is in the evaluation.    
\end{itemize}

\section{Architecture Design and Implementation}

\subsection{Overall architecture}
\begin{figure}[H]
    \centering
    \includegraphics[width=0.75\linewidth]{ziggsDB.png}
    \caption{Database structure}
    \label{fig:database}
\end{figure}


\begin{figure}[H]
    \centering
    \includegraphics[width=1\linewidth]{architecture.png}
    \caption{System Architecture}
    \label{fig:system_architecture}
\end{figure}

\section{Modules Overview}

\subsection{Directory organization Backend}

\begin{table}[H]
\centering
\renewcommand{\arraystretch}{1.5}
\begin{tabular}{|m{2.6cm}|m{3.5cm}|m{1.6cm}|}
\hline
\textbf{Directory} & \textbf{File Names} & \textbf{Module} \\ \hline

Ziggs\_Backend/ src/main/java/ com/ziggs/ ziggs\_backend/ config & 
AwsS3Config.java \newline 
FirebaseConfig.java \newline 
WebConfig.java & 
Config \\ \hline

Ziggs\_Backend/ src/main/java/ com/ziggs/ ziggs\_backend/ controller & 
AlertController.java \newline 
AuthController.java \newline 
DeviceController.java \newline 
FileUploadController.java \newline 
HouseController.java \newline 
SecureController.java \newline 
SoundDataController.java \newline 
UserController.java& 
Controller \\ \hline

Ziggs\_Backend/ src/main/java/ com/ziggs/ ziggs\_backend/ dto & 
AlertDTO.java \newline 
DeviceDTO.java \newline 
LoginRequestDTO.java \newline 
S3FileDTO.java \newline 
SecureDTO.java \newline 
SoundDataDTO.java \newline 
UserDTO.java \newline 
UserIdDTO.java & 
DTO \\ \hline

Ziggs\_Backend/ src/main/java/ com/ziggs/ ziggs\_backend/ entity & 
Alert.java \newline 
Device.java \newline 
House.java \newline 
SoundData.java \newline 
User.java& 
Entity \\ \hline

Ziggs\_Backend/ src/main/java/ com/ziggs/ ziggs\_backend/ repository & 
AlertRepository.java \newline 
DeviceRepository.java \newline 
HouseRepository.java \newline 
SoundDataRepository.java \newline 
UserRepository.java & 
Repository \\ \hline

Ziggs\_Backend/ src/main/java/ com/ziggs/ ziggs\_backend/ security & 
FirebaseAuthInterceptor.java \newline 
PasswordHasher.java \newline 
SecurityConfig.java & 
Security \\ \hline

Ziggs\_Backend/ src/main/java/ com/ziggs/ ziggs\_backend/ service & 
AlertService.java \newline 
DeviceService.java \newline 
HouseService.java \newline 
S3Service.java \newline 
SoundDataService.java \newline 
UserService.java & 
Service \\ \hline

Ziggs\_Backend/ src/main/resources & 
application.yml \newline 
serviceAccountKey.json & 
Resources \\ \hline

Ziggs\_Backend & 
Dockerfile & 
Docker \\ \hline

Ziggs\_Backend & 
pom.xml & 
Build \\ \hline


\end{tabular}
\caption{Directory Structure and Modules for Backend}
\end{table}


The application is divided into several modules, each with a specific role and responsibility to ensure clarity, modularity, and scalability.

\vspace{0.5cm}

\textbf{Controller:}  

This module manages incoming HTTP requests and serves as the entry point for users to interact with the application. It defines endpoints for features such as authentication, notifications, user management, and feedback handling. Requests received by the controllers are passed to the appropriate service layer for processing. By decoupling request handling from business logic, it ensures clarity, modularity, and easier debugging.

\vspace{0.5cm}

\textbf{Service:}  

This module contains the core business logic of the application. It processes data received from the controllers, applies business rules, and handles complex operations before interacting with the repositories for data storage or retrieval. For example, the AlertService validates incoming alerts before storing them, and the UserService manages user-related operations like registration and profile updates. This separation ensures maintainability, scalability, and flexibility for adapting to future requirements.

\vspace{0.5cm}

\textbf{Repository:}  

The repository module handles data persistence and retrieval. It interfaces directly with the database using JPA/Hibernate, abstracting the complexity of database queries. Each repository corresponds to a specific entity, such as \texttt{UserRepository} for user data or \texttt{AlertRepository} for alerts, ensuring efficient data access and consistency. This design provides standardized and reusable methods for database operations.

\vspace{0.5cm}

\textbf{Entity:}  

The entity module defines the application's data models. It maps Java objects to database tables using JPA annotations, providing a clear and structured way to represent data. Entities such as \texttt{User}, \texttt{Alert}, and \texttt{FeedbackTicket} define the schema and relationships between different data points. This abstraction ensures database interactions remain intuitive and aligned with the business logic.

\vspace{0.5cm}

\textbf{DTO (Data Transfer Object):}  

This module structures data for specific use cases, such as API requests and responses. DTOs like \texttt{UserDTO} and \texttt{LoginRequestDTO} are used to simplify communication between layers by encapsulating only the required fields. They improve performance by reducing unnecessary data transfer and ensure consistent data formats across endpoints.

\vspace{0.5cm}

\textbf{Configuration:}  

The configuration module manages external library integrations and application settings. It includes files like:

\begin{itemize}
    \item \textbf{FirebaseConfig}: Sets up Firebase authentication.
    \item \textbf{WebConfig}: Defines interceptors and CORS policies.
    \item \textbf{AwsS3Config}: Configures AWS S3 integration by injecting access keys, region, and creating an Amazon S3 client bean for seamless interaction with S3 buckets.
\end{itemize}

This module ensures scalability, adaptability, and easy customization of application behavior without modifying the core logic.


\vspace{0.5cm}

\textbf{Resources:}  

This directory contains essential configuration files required for the application to run properly. It includes \texttt{application.yml} for defining environment-specific settings (e.g., database connections) and Firebase credentials for integrating authentication services as well as S3 connection. These files centralize setup details and streamline environment configuration.

\vspace{0.5cm}

\textbf{Docker:}  

The Docker module provides the steps necessary to containerize the application. The \texttt{Dockerfile} defines how the application is built and run in a Docker container, ensuring consistency across development, testing, and production environments. It simplifies deployment, allows better resource isolation, and enables scalability through container orchestration tools.

\vspace{0.5cm}

\textbf{Build:}  

The \texttt{pom.xml} file defines the project's dependencies, plugins, and build structure. It ensures compatibility by managing library versions and provides an efficient way to build, test, and deploy the application. This file acts as the foundation for the Maven-based build process.

\subsection{Directory organization Data and AI}
\begin{table}[H]
\centering
\renewcommand{\arraystretch}{1.5}
\begin{tabular}{|m{2.6cm}|m{3.5cm}|m{1.6cm}|}
\hline
\textbf{Directory} & \textbf{File Names} & \textbf{Module} \\ \hline

AI\_Model/ & 
Sound\_Preprocessing.ipynb \newline 
Pytorch\_Supervised\_Model.ipynb & 
Data Processing / Model Creation \\ \hline

AWS\_Backend\_Model/ & 
app.py \newline 
app\_api.py \newline 
model.py \newline 
CNN\_RegDrop.pt & 
S3 Monitoring and Model Execution / Legacy API \\ \hline

AWS\_Backend\_Model/ & 
dockerfile & 
Deployment Configuration \\ \hline

\end{tabular}
\caption{Directory Structure and Modules for AI Workflow}
\label{table:workflow_structure}
\end{table}

\subsubsection{Data Processing / Model}
This module preprocesses audio data into mel spectrograms, trains a CNN model to classify audio anomalies, and evaluates the model's performance. Detailed explanations can be found in section \ref{AI Impls}.

\textbf{Key Components}:
\begin{itemize}
    \item \textbf{Data Preparation}: Converts audio into mel spectrograms and structures datasets.
    \item \textbf{CNN Architecture}: Implements convolutional layers, dropout, and batch normalization for classification.
    \item \textbf{Training and Validation}: Trains the model with cross-entropy loss and evaluates metrics such as precision and accuracy.
\end{itemize}

\subsubsection{S3 Monitoring and Model Execution}
This module automates anomaly detection by monitoring an S3 bucket for audio files, processing them with a CNN model, and sending notifications if anomalies are detected. 

\textbf{Initial Design}: 
Originally, an API Gateway with AWS Lambda was implemented to provide predictions via a \texttt{/predict} endpoint. This design worked well for individual requests, as shown in Figure~\ref{fig:postman_api}. However, maintaining multiple API endpoints across different backends is too complex for a system where data processing could be centralized.  

\textbf{Current Design}: 
The job-based system in \texttt{app.py} (Work in progress):
\begin{itemize}
    \item Monitors the \texttt{input-data/} folder in S3.
    \item Processes files with a preloaded CNN model (\texttt{CNN\_RegDrop.pt}).
    \item Sends notifications for anomalies and moves processed files to \texttt{processed-data/}.
\end{itemize}

\subsubsection{Dockerfile}
This Dockerfile was created for the initial design where the AI model was deployed via an API gateway and AWS Lambda. It uses the AWS Lambda Python runtime base image and is optimized for small size by installing only the necessary dependencies such as PyTorch, Flask and \texttt{apig-wsgi}. 
The application code (\texttt{app.py}, \texttt{model.py}) and the trained model (\texttt{CNN\_RegDrop.pt}) are included in the container. The resulting image is pushed to Amazon Elastic Container Registry (ECR) for deployment in AWS Lambda.

\subsection{Directory Organization Front-End}

\begin{table}[H]
\centering
\renewcommand{\arraystretch}{1.5}
\begin{tabular}{|m{2.6cm}|m{3.5cm}|>{\raggedright\arraybackslash}m{1.6cm}|}
\hline
\textbf{Directory} & \textbf{File Names} & \textbf{Module} \\ \hline

Vue\_Frontend/src/  & 
App.vue \newline
main.js & 
Initialization \\ \hline

Vue\_Frontend/src/ router & 
index.js \newline
routes.js & 
Routing \\ \hline

Vue\_Frontend/src/ assets/ & 
audio/*.wav \newline
css/*.css \newline
fonts/*.woff \newline
fonts/*.woff2 \newline
fonts/*.ttf \newline
fonts/*.eot \newline
fonts/*.svg \newline
img/*.png \newline
img/*.jpg \newline
img/*.svg \newline
img/*.gif \newline
sass/*.scss & 
Assets \\ \hline

Vue\_Frontend/src/  components/ & 
Button.vue \newline
Dropdown.vue \newline
FormCompenent.vue \newline
ziggTable.vue \newline
DragModal.vue \newline
Inputs/formGroupInput.vue \newline
Inputs/LabelledInput.vue \newline
index.js & 
Vue Components \\ \hline

\end{tabular}
\end{table}

\begin{table}[H]
\centering
\renewcommand{\arraystretch}{1.5}
\begin{tabular}{|m{2.6cm}|m{3.5cm}|m{1.6cm}|}
\hline
\textbf{Directory} & \textbf{File Names} & \textbf{Module} \\ \hline

Vue\_Frontend/src/ components/Cards & 
BasicCard.vue \newline
Card.vue \newline
ChartCard.vue \newline
StatsCard.vue \newline
TableCard.vue & 
Card Vue Components\\ \hline

Vue\_Frontend/src/ components/ SidebarPlugin & 
SidebarLink.vue \newline
SideBar.vue \newline
MovingArrow.vue \newline
index.js & 
Sidebar Vue Components \\ \hline

Vue\_Frontend/src/ layout & 
AlarmTab.vue \newline
FooterComponent.vue \newline
GlobalyLayout.vue \newline
HeaderComponent.vue \newline
ModalLayout.vue & 
Layouts \\ \hline

Vue\_Frontend/src/ layout/dashboard & 
AlarmHandlingModal.vue \newline
AlarmModal.vue \newline
Content.vue \newline
ContentFooter.vue \newline
DashboardLayout.vue \newline
MobileMenu.vue \newline
NotificationTable.vue \newline
TopNavbar.js & 
Dashboard Layout \\ \hline

Vue\_Frontend/src/ pages & 
Dashboard.vue \newline
Devices.vue \newline
Login.vue \newline
Maps.vue \newline
NotFoundPage.vue \newline
Notificiations.vue \newline
Recording.vue \newline
Signin.vue \newline
TableList.vue \newline
Typography.vue \newline
UserProfile.vue  \newline
Notifications/ NotificationTemplate.vue &
Pages \\ \hline

Vue\_Frontend/src/ pages/UserProfile & 
components/edit-house-infos-modal.vue \newline
components/edit-house-members-modal.vue \newline
HouseInfos.vue \newline
HouseMember.vue \newline
UserCard.vue & 
User Profile Components \\ \hline

Vue\_Frontend/src/ plugins & 
globalComponents.js \newline
ziggDashboard.js \newline
globalDirectives.js & 
Plugins\& Global Functions\\ \hline

\end{tabular}
\caption{Directory Structure and Modules for Front-end}
\end{table}

\vspace{0.5cm}

\textbf{Initialization:}

This module is in charge of the initialization of the whole project. It loads the dashboard with Vue and activate the routing part.

\vspace{0.5cm}

\textbf{Routing:}

The routing files create an instance of the Router class and save all the routes (URIs) of our project. It links the routes with the wanted components. The routing is essential to allow the website to have multiple pages.

\vspace{0.5cm}

\textbf{Assets:}

The Assets section is used for no code files such as style sheets and images. The files of the Assets folder are not fixed and may be deleted or created during its lifetime.

\vspace{0.5cm}

\textbf{Vue Components:}

Vue.JS works with components, which are pieces of HTML/CSS and a script language you can implement with other components to create pages. This module Vue Components contains the components used in our pages. Some components use other components to work, these components are grouped in sub-folders.

\vspace{0.5cm}

\textbf{Card Vue Components:}

This section is about the sub-folder "Cards" of the Components module. It contains different kind of Cards components and it is where new Cards components should be stored.

\vspace{0.5cm}

\textbf{Side Bar Vue Components:}

This section is about the sub-folder "SidebarPlugin" of the Components module. It contains the Sidebar component and its sub-components as well.

\vspace{0.5cm}

\textbf{Layouts:}

This module gathers the Layouts components. A layout is a component which defines the position of elements. A layout can be used for different pages. This module can contain sub-folders for layouts which need specific components. These sub-folders contain the layout and its components.

\vspace{0.5cm}

\textbf{Dashboard Layout:}

This sub-folder of the Layouts module is meant to store the Dashboard component and its sub-components.

\vspace{0.5cm}

\textbf{Pages:}

This Pages module contain the pages components of the project. These components may use or complete a layout with more specific information. Pages components are basically the components that are displayed to the user. Some pages may use its own components, in this case the page and its components can be put in the same folder.

\vspace{0.5cm}

\textbf{User Profile Components:}

This sub-folder of the Pages module stores the User Profile page and its sub-components in a folder.

\vspace{0.5cm}

\textbf{Plugins \& Global functions:}  

The Plugins module is used to import external libraries and globalize their components to the whole project. It is also used to make some functions global in the project such as the request to execute HTTP Requests.

\vspace{0.5cm}
\textbf{Login/Create Account Page}

This module handles user authentication and account creation using [specify technology or framework, e.g., Firebase, OAuth]. It supports multiple authentication methods, including email/password and social media logins. Upon successful login or account creation, the user is redirected to the dashboard. User session management is implemented to maintain login states across visits. The authentication process is asynchronous, with progress animations to enhance the user experience during the wait.
\vspace{0.5cm}

\textbf{Household Management Page}

On the Household Management page, users can add and manage household members. This feature allows users to define roles and permissions for each member concerning device access and data visibility. The interface provides forms to input member information and uses [technology or method, e.g., AJAX calls] to submit this data to the server. Changes are updated in real-time using [state management technique, e.g., Redux or Context API], ensuring all household data across the platform remains synchronized.
\vspace{0.5cm}

\textbf{Devices Page}

The Devices page enables users to register and manage devices equipped with microphones. It includes a form to enter device details and a mechanism to connect these devices to the user’s network. Each device sends audio data to the backend for processing. The page provides real-time feedback on the device’s status, such as connectivity and recording state, and uses asynchronous requests to handle registration and updates without page reloads.
\vspace{0.5cm}

\textbf{Dashboard Page}

The Dashboard serves as a central hub for users to get an overview of their household and device settings. It displays aggregate data from all connected devices, including audio monitoring status and event logs. The dashboard updates dynamically, reflecting changes in real-time through web sockets or polling mechanisms. Widgets and graphical representations (charts, graphs) provide a user-friendly interface to monitor household activities and device statuses.
\vspace{0.5cm}

\textbf{Popup Alerts}

Popup alerts are implemented across all pages to notify the user of any detected abnormalities by the AI in the backend. These alerts are triggered in real-time and are designed to catch the user’s attention without disrupting ongoing tasks. The alerts provide concise information about the nature of the abnormality and prompt the user for immediate actions if necessary. Integration with the backend AI uses [specific technologies, protocols], ensuring timely and relevant notifications.
\vspace{0.5cm}

\textbf{Backend Integration and AI Processing}

This module describes the interaction between the frontend and the AI algorithms running on the server. The backend receives audio data from devices, processes it using machine learning models to detect abnormalities, and sends alerts back to the frontend. API endpoints are designed for efficient data transmission and security. Details include the use of RESTful services or real-time communication protocols like WebSockets for continuous data flow.

\section{Use Cases}

\subsection{Backend}

The following examples illustrate some use cases for endpoints within the web application. These are just a few among many others, meant to demonstrate how various functionalities are integrated into the system to handle different types of user requests.

\begin{figure}[H]
    \centering
    \includegraphics[width=0.75\linewidth]{post_use_case.png}
    \caption{Endpoint for creating a user. The user enters his information such as username, password, address... The user information is then stored into our relational database.}
\end{figure}


\begin{figure}[H]
    \centering
    \includegraphics[width=0.75\linewidth]{get_use_case.png}
    \caption{ Endpoint for retrieving user profile data. This endpoint allows the web app to fetch detailed information about a user, such as their name, email...}
\end{figure}

\begin{figure}[H]
    \centering
    \includegraphics[width=0.75\linewidth]{token_use_case.png}
    \caption{ Endpoint for user login. Upon successful login, the system generates a custom token that grants access to restricted endpoints based on the user's role, while also maintaining the session.}
\end{figure}


These use cases represent just a fraction of the possible interactions within the application, each designed to handle specific tasks and improve user experience through seamless API communication.

\subsection{AI Model Backend}

Figure \ref{fig:postman_api} illustrates the use of an API endpoint for anomaly detection in audio data. A POST request is sent to the endpoint hosted on AWS API Gateway, which provides audio input data in JSON format. 

The API processes the input using the provided AI model and returns the following response
\begin{itemize}
    \item \textbf{Anomaly Score}: A numeric value indicating the probability that the input is anomalous.
    \item \textbf{Predicted Class}: The classification result, where 0 means that no anomaly was detected.
    \item \textbf{Predicted Probability}: The confidence value for the predicted class.
\end{itemize}

\begin{figure}[H]
    \centering
    \includegraphics[width=0.75\linewidth]{API Model.png}
    \caption{Postman Test of API Gateway Design}
    \label{fig:postman_api}
\end{figure}

Figure \ref{fig:s3} shows the \texttt{input-data/} folder in an Amazon S3 bucket. This folder is continuously monitored by the new job-based approach implemented in \texttt{app.py}. 

\begin{figure}[H]
    \centering
    \includegraphics[width=0.8\linewidth]{s3.png}
    \caption{S3 Bucket that is being monitored}
    \label{fig:s3}
\end{figure}

\subsection{Front-End}

Figure \ref{fig:user_dashboard} shows the user dashboard.
To access this page, the user just need to login. It is composed of a side menu, the main part and a header. By default, the main part displays the last alerts and notifications. The user can navigate through the dashboard with the side menu.

\begin{figure}[H]
    \centering
    \includegraphics[width=0.8\linewidth]{user_dashboard_use_case.png}
    \caption{The user dashboard main page}
    \label{fig:user_dashboard}
\end{figure}

Figure \ref{fig:household} shows the household section during editing.
To access the household page of the dashboard, the user has to click on the "Household" button in the side menu. At this point, the user can see the household members and information. The "Edit Profile" button will open the modal you can see on the screenshot.
When the form is completed and submitted, the household information are updated.

\begin{figure}[H]
    \centering
    \includegraphics[width=0.8\linewidth]{edit_household_use_case.png}
    \caption{Editing household's information}
    \label{fig:household}
\end{figure}

Figure \ref{fig:devices} shows the "Devices" section of the dashboard.
This section can be accessed by clicking on the "Devices" button on the side menu. From this page users are able to add new devices linked to their home or remove the existing devices.

\begin{figure}[H]
    \centering
    \includegraphics[width=0.8\linewidth]{devices_page_use_case.png}
    \caption{Devices dashboard}
    \label{fig:devices}
\end{figure}

Figure \ref{fig:host_device} shows the popup to add the device you are on to your device list. It can be accessed from the device dashboard, clicking on "Add a new device" and "Add this device".
Once validated, the recording page opens in a new tab.

\begin{figure}[H]
    \centering
    \includegraphics[width=0.8\linewidth]{add_host_device_use_case.png}
    \caption{Adding host device popup}
    \label{fig:host_device}
\end{figure}

Figure \ref{fig:lg_devices} shows the popup to add other LG connected devices. It can be accessed from the device dashboard, clicking on "Add a new device" and "Add LG Device".

\begin{figure}[H]
    \centering
    \includegraphics[width=0.8\linewidth]{add_LG_device_use_case.png}
    \caption{Adding LG Device popup}
    \label{fig:lg_devices}
\end{figure}

Figure \ref{fig:recording} shows the recording web page. This page works only on devices that have been linked to an account. Once the start button has been clicked and the microphone allowed, this page records the surrounding sounds continuously. Records are cut in 5 seconds pieces that are sent to the Back-end server for analysis by the AI Model.

\begin{figure}[H]
    \centering
    \includegraphics[width=0.8\linewidth]{recording_screen.png}
    \caption{Sound Recording screen}
    \label{fig:recording}
\end{figure}


\section{Discussion}
During the project, we faced several challenges. One notable difficulty was the lack of consistent organization. Despite initial plans, there were moments where deadlines crept up on us, leading to a rushed approach to certain tasks. This was particularly evident in the integration phase, where the frontend development lagged significantly behind the backend and AI model progress. This disconnect meant that the final stages required an intense push to bring everything together seamlessly. While we managed to pull it off, it was a stressful experience that could have been avoided with better coordination and a more synchronized development timeline.
\newline 

On a positive note, these challenges provided valuable learning opportunities. We gained insights into the importance of regular communication, task prioritization, and maintaining alignment across different project components. Working under pressure also highlighted our ability to adapt and problem-solve effectively as a team. In the end, we are proud of what we accomplished. The experience not only improved our technical skills but also strengthened our ability to navigate the complexities of collaborative projects. While there is certainly room for improvement, we are leaving this project with a sense of achievement and lessons that will undoubtedly guide us in future endeavors.

\end{document}





